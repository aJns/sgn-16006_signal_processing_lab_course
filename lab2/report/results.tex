\section{Results}
Using both correlation detection and Goertzel-algorithm we calculated the values of the frequencies in the input signal. Then, we compared these values to a threshold. If the value is greater thn this threshold, a corresponding LED will light. We used the following MATLAB code to test the thresholds. The first part of the code goes through alla the frequencies, lighting one LED at a time. The others test on one or two frequencies.

\begin{lstlisting}
t = 0:1/2000:5;
x = chirp(t, 0, 5, 1000);
soundsc(x, 2000);

%%
fs = 8192;
f = 697; % This is the studied frequency
t = 0:1/fs:5;
x = sin(2*pi*t*f);
soundsc(x, fs);

%%
f = 697; % This is the studied frequency
n = 1:8192;
x = sin(2*pi*n*f / 8192);
soundsc(x);

%%
fs = 8192;
f1 = 770;
f2 = 1336;
t = 0:1/fs:5;
x = sin(2*pi*t*f1)+sin(2*pi*t*f2);
soundsc(x, fs);

%%
fs = 8192;
f1 = 852;
f2 = 1209;
t = 0:1/fs:5;
x = sin(2*pi*t*f1)+sin(2*pi*t*f2);
soundsc(x, fs);
\end{lstlisting}

Using these tests we were able to pinpoint a threshold which would be able to distinguish the different frequencies. With sampling frequency of 3000HZ and period of period 333µs the threshold is 8 in both methods.