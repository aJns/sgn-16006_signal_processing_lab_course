\section{Methods}
In bilinear interpolation the color value of the pixels is calculated from the color value of neighboring pixels with same color i.e sum the values of neighboring pixels and divide it by the amount of neighboring pixels. However, the pixels at the edges of the picture don't always have neighboring pixels with color value or don't have enough of them. To correct this we can just ignore the border pixels, resulting in smaller image, or as we did in the assignment and pad the color arrays with zeros. The color values of the layers are in different pixels, so every layer needs an unique function to calculate the value of an unknown pixel.  Green has the most known color values, and every unknown pixel borders four known pixels, so one function can calculate all values of the green layer.

Only every other row of red and green layers has colorvalues, which has to be taken into account when designing these functions. The rows which have no colorvalues compute the value differently, depending on how many known values border the pixel, and use either four or two values.

Colorvalues of the pixels are calculated from the padded color array and added to the unpadded one. After all values of all colorarrays are calculated, the arrays are combined into RGB image.