\section{Results and conclusions}

\subsection{Results}
In this assigment, the overall classification accuracy didn't improve
significantly by increasing the k value. The accuracy with k=1 was 72.1\%, and
with k=5 73.5\% which is only a 1.4 percentage point increase. In many test
runs the accuracies were within <0.1\% of each other.

Looking at the confusion matrices, one interesting thing we see is that with
k=5 there's less confusion overall, except for classes 11, 8 and to a lesser
extent 6, which are mislabeled as classes 8, 12 and 4 respectively. The rate of
correct labels is improved with k=5, most noticeably with class 12.

\subsection{Conclusions}
In realistic situations nearest neighbor isn't the greatest classifier because
a) there are more accurate classifiers and b) it's slow at classifying. It's
pretty much the slowest classifier at classifying, because it has to go through
the whole training data each time it classifies something. Contrast it with
other classifiers like random forests, which can take a while to train but
whose classification is pretty much instantaneous.

\subsection{Feedback}
The instructions were easy enough to follow. I don't know if I learned anything
super useful, as framing and nearest neighbor classifiers are pretty basic
info. It took us about 4--5 hours to do the whole assignment and an additional 2
hours for the corrections.
